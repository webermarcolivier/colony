% NEW COMMANDS
\newcommand{\sstitre}[1]{\noindent\textbf{#1}}	% write a subtitle for a section
\newcommand{\super}[1]{\textsuperscript{#1}}	% text superscript
\newcommand{\sub}[1]{\textsubscript{#1}}	% text subscript

\newcommand{\field}[1]{\mathbb#1}	% write ensembles (R,C,Z,N) with a special font
\newcommand{\f}{\frac}
\newcommand{\abs}[1]{\lvert#1\rvert}		% absolute value
\newcommand{\norm}[1]{\lVert#1\rVert}		% norm
\newcommand{\vv}[1]{\vec#1}			% vector
\newcommand{\m}[1]{\mathbf#1}			% matrix
\newcommand{\op}[1]{\mathbf{\widehat{#1}}}	% operator
\newcommand{\ket}[1]{\left|#1\right.\rangle}	% quantum mechanics ket
\newcommand{\bra}[1]{\langle\left.#1\right|}	% quantum mechanics bra
\newcommand{\braket}[2]{\langle\left.#1\right| #2 \rangle}	% quantum mechanics braket product
\DeclareMathOperator{\grad}{\overrightarrow{\textrm{grad}}}     % gradient
\DeclareMathOperator{\erf}{\textrm{erf}}	% error function
\DeclareMathOperator{\Var}{\textrm{Var}}	% variance

\newcommand{\degree}{\degres}			% celsius degrees !\degres needs french package!
\newcommand{\ee}[2]{\cdot 10^{#1}\ \text{#2}}	% to write the 10 exponent after a numerical results + units in textstyle. For example, 2x10^6 m/s
\newcommand{\eval}[2]{\left.#1\right|_{#2}}	% right vertical bar to indicate `evaluate f(t) at t=0'
\newcommand{\avg}[1]{\langle#1\rangle}		% statistical average < >  (look better than < and > characters)
\newcommand{\eps}{\varepsilon}			% common epsilon greek letter
\newcommand{\s}[2]{#1_{\text{#2}}}		% write textstyle subscript in math mode (use it if more than one character in the subscript)

% commands for derivatives and partial derivatives
%\newcommand{\dd}[2]{\frac{\mathrm{d}#1}{\mathrm{d}#2}}		% textstyle 'd' (no italic) version
%\newcommand{\ddd}[2]{\frac{\mathrm{d}^2#1}{\mathrm{d}#2^2}}    % textstyle 'd' (no italic) version
\newcommand{\dd}[2]{\frac{d#1}{d#2}}				% first derivative
\newcommand{\ddd}[2]{\frac{d^2#1}{d#2^2}}			% second derivative
\newcommand{\ddp}[2]{\frac{\partial#1}{\partial #2}}		% first partial derivative
\newcommand{\dddp}[2]{\frac{\partial^2#1}{\partial #2^2}}	% second partial derivative
\newcommand{\dx}[1]{\text{d}#1}					% differential element dx, dA, ds with 'd' textstyle (no italic)
\newcommand{\dt}[1]{\overset{\textstyle{.}}{#1}}		% dot style time derivative